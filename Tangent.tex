\documentclass[12pt]{report}

\usepackage{amsmath, amssymb, geometry}
\usepackage{setspace}
\usepackage{titlesec}
\usepackage{graphicx}
\geometry{margin=1in}
\setstretch{1.5}

\titleformat{\chapter}{\centering\Huge\bfseries}{\thechapter}{1em}{}

\begin{document}
	
	\begin{titlepage}
		\centering
		{\Large \textbf{APPLICATION TO THE GALOIS GROUPS OF POLYNOMIALS}}\\[2cm]
		
		{\large A Final Year Project}\\[1cm]
		
		Submitted in partial fulfillment of the requirements for the award of\\
		Bachelor of Science Degree in Mathematics\\[2cm]
		
		\vfill
		
		\textbf{Musa Ojonugwa James}\\
		SCI22MTH019\\[1cm]
		
		Department of Mathematics\\
		Federal University Lokoja\\
		2026
		
	\end{titlepage}
	
	\chapter*{Abstract}
	
	Galois theory provides a framework for understanding polynomial equations through the symmetry properties of their roots. Rather than solving equations directly, the theory studies automorphisms of field extensions that preserve algebraic relations among roots. This project investigates computational approaches to determining Galois groups of polynomials and interpreting their algebraic structures. Emphasis is placed on discriminants, root permutations, and group classification for polynomials of degree up to five. A computational interface is also developed to support the analysis and visualization of polynomial symmetries.
	
	\chapter{Introduction}
	
	\section{Background of the Study}
	
	The study of polynomial equations has been a central topic in algebra for centuries, with mathematicians seeking general methods to find solutions. While formulas exist for solving quadratic, cubic, and quartic equations using radicals, no general formula applies to quintic or higher-degree equations. This limitation was resolved through the development of Galois theory by Évariste Galois, who demonstrated that the solvability of a polynomial depends on the structure of the symmetry group of its roots rather than on the roots themselves. Galois theory establishes a deep connection between polynomial roots, field extensions, and permutation groups, providing a framework to determine whether a polynomial is solvable by radicals. By studying the automorphisms of a polynomial’s splitting field, one can classify the symmetries of its roots and identify the corresponding Galois group.	With the advent of modern computational tools, it is now possible to explore these concepts numerically and symbolically. Computational approaches allow for the efficient evaluation of discriminants, the approximation of roots, and the identification of likely Galois groups. This combination of theoretical and computational methods makes it feasible to analyze polynomials of higher degree and provides an accessible way for students and researchers to understand the structure and solvability of algebraic equations.
	
	
	\section{Statement of the Problem}
	
	Determining the Galois group of a polynomial is essential for understanding its solvability and the underlying algebraic structure of its roots. However, as the degree of a polynomial increases, manual computation of roots, discriminants, and automorphisms becomes increasingly complex and error-prone. The relationships among roots can be intricate, and identifying the corresponding symmetry groups often requires advanced theoretical analysis that is not easily accessible to students or researchers without specialized tools. This project addresses the need for computational methods that can efficiently analyze polynomial roots, compute discriminants, and suggest likely Galois groups, thereby making the study of solvability and symmetry more practical and accessible.
	
	
	\section{Aim and Objectives}
The aim of the project is to develop computational methods for analyzing the Galois groups of polynomials and to explore how these methods can be used to determine the solvability and symmetry properties of algebraic equations. To achieve our aim, the following objectives have been stated.
	\begin{enumerate}
		\item[(i.)] Explain the theoretical foundation of Galois groups and field extensions.
		\item[(ii.)] Compute discriminants and analyze root structures.
		\item[(iii.)] Determine Galois groups of selected polynomials.
		\item[(iv.)] Implement computational tools for symmetry analysis.
		\item[(v.)] Provide a graphical interface for polynomial analysis.
	\end{enumerate}
	
	\section{Scope and Limitation of the Study}
	
The scope of the project is focused on the study and computational analysis of polynomials with rational coefficients, specifically those of degrees two through five. It includes the computation of roots, evaluation of discriminants, and identification of likely Galois groups using numerical and algebraic methods. The project also incorporates a graphical interface for inputting polynomials and visualizing results, making it applicable for teaching, learning, and basic research purposes. Emphasis is placed on demonstrating the principles of Galois theory, exploring solvability by radicals, and linking abstract algebraic concepts to computational techniques. The limitations of the study arise from both theoretical and computational constraints. Exact determination of Galois groups becomes increasingly difficult for polynomials of degree higher than five due to the complexity of their symmetry structures. Numerical methods may introduce rounding errors in root approximations, which can affect the precision of discriminants and group identification. The study does not cover all possible polynomials over arbitrary fields, nor does it implement full algebraic methods for complete group classification beyond quintic polynomials. Additionally, while the computational tool provides heuristic identification of Galois groups, it does not replace formal algebraic proofs.

	
	\section{Significance of the Study}
	
The study is significant because it bridges abstract algebraic theory and practical computation, making the concepts of Galois theory more accessible and understandable. By providing computational methods to analyze polynomial roots, discriminants, and symmetries, it enhances the teaching and learning of advanced algebra and supports students in visualizing and exploring algebraic structures. The project also demonstrates the practical relevance of Galois theory in areas such as cryptography, coding theory, and symbolic computation, where understanding field extensions and group symmetries is essential. Furthermore, the development of a computational tool allows for efficient exploration of polynomials, enabling researchers and learners to investigate solvability and group properties without relying solely on manual calculations. This contributes to both educational enrichment and the application of algebraic principles in computational mathematics.

	
	\section{Definition of Terms}
	
	\begin{enumerate}
\item	\textbf{Polynomial:} An algebraic expression consisting of variables and coefficients combined using addition and multiplication.
	
\item	\textbf{Field:} A set in which addition, subtraction, multiplication, and division are defined and behave normally.
	
\item	\textbf{Field Extension:} A larger field containing a smaller field and additional elements such as polynomial roots.
	
\item	\textbf{Splitting Field:} The smallest field containing all roots of a polynomial.
	
\item	\textbf{Automorphism:} A structure-preserving map from a field to itself that fixes the base field.
	
\item	\textbf{Galois Group:} The group of automorphisms of a polynomial’s splitting field that preserve algebraic relations among its roots.
	
\item	\textbf{Discriminant:} A quantity derived from a polynomial that provides information about root multiplicity and symmetry.
	
\item	\textbf{Solvable Group:} A group whose structure allows a polynomial to be solved using radicals.
	\end{enumerate}
	
	\chapter{Methodology}
	
	This project uses numerical root computation and discriminant analysis to infer symmetry properties of polynomial roots. The discriminant provides information about repeated roots and symmetry, while root permutations suggest the structure of the Galois group. For example, Consider the quadratic polynomial
	\[
	x^2 - 2 = 0.
	\]
	
	\section*{Solution}
	
	\subsection*{Step 1: Solve for $x$}
	
	Isolate $x^2$:
	\[
	x^2 = 2
	\]
	
	Take the square root of both sides:
	\[
	x = \pm \sqrt{2}.
	\]
	
	Hence, the roots are
	\[
	x_1 = \sqrt{2}, \quad x_2 = -\sqrt{2}.
	\]
	
	\subsection*{Step 2: Splitting Field}
	
	The splitting field is the smallest field containing both roots:
	\[
	\mathbb{Q}(\sqrt{2}).
	\]
	
	\subsection*{Step 3: Automorphisms}
	
	The automorphisms of the splitting field that fix $\mathbb{Q}$ are:
	
	\begin{enumerate}
		\item Identity: $\sqrt{2} \mapsto \sqrt{2}$
		\item Conjugation: $\sqrt{2} \mapsto -\sqrt{2}$
	\end{enumerate}
	
	\subsection*{Step 4: Galois Group}
	
	The set of automorphisms forms a group under composition:
	\[
	\text{Gal}(x^2 - 2) = \{ \text{id}, \text{conjugation} \} \cong C_2
	\]
	
	\subsection*{Step 5: Solvability by Radicals}
	
	Since the Galois group has order $2$ and is abelian, it is solvable. Therefore, the quadratic polynomial
	\[
	x^2 - 2 = 0
	\]
\noindent	is solvable by radicals. The roots of the polynomial are $\sqrt{2}$ and $-\sqrt{2}$, the splitting field is $\mathbb{Q}(\sqrt{2})$, and the Galois group is $C_2$. This confirms that the equation is solvable by radicals.
\section*{Example 2: Cubic Polynomial}

Consider the cubic polynomial
\[
x^3 - 2 = 0.
\]

\subsection*{Step 1: Roots}

The roots are
\[
x_1 = \sqrt[3]{2}, \quad x_2 = \sqrt[3]{2}\,\omega, \quad x_3 = \sqrt[3]{2}\,\omega^2,
\]
where 
\(\omega = e^{2\pi i/3}\) is a primitive cube root of unity.

\subsection*{Step 2: Splitting Field}

The splitting field is
\[
\mathbb{Q}(\sqrt[3]{2}, \omega).
\]

\subsection*{Step 3: Discriminant}

For a cubic polynomial \(x^3 + px + q\), the discriminant is
\[
\Delta = -4p^3 - 27q^2.
\]

Here, \(p = 0\) and \(q = -2\), so
\[
\Delta = -27(-2)^2 = -108.
\]

\subsection*{Step 4: Galois Group}

Since the discriminant is negative and not a perfect square in \(\mathbb{Q}\), the Galois group is
\[
S_3.
\]

\subsection*{Step 5: Solvability by Radicals}

The symmetric group \(S_3\) is solvable; therefore, the cubic polynomial is solvable by radicals.

\section*{Example 3: Quartic Polynomial}

Consider the quartic polynomial
\[
x^4 - 5x^2 + 6 = 0.
\]

\subsection*{Step 1: Factorization}

Let \(y = x^2\). Then
\[
y^2 - 5y + 6 = 0 \implies (y-2)(y-3)=0.
\]

\subsection*{Step 2: Roots}

Thus, the roots are
\[
x = \pm \sqrt{2}, \quad x = \pm \sqrt{3}.
\]

\subsection*{Step 3: Splitting Field}

The splitting field is
\[
\mathbb{Q}(\sqrt{2}, \sqrt{3}).
\]

\subsection*{Step 4: Galois Group}

The automorphisms can independently send
\(\sqrt{2} \to \pm \sqrt{2}\) and \(\sqrt{3} \to \pm \sqrt{3}\).  
Therefore, the Galois group has order 4 and is isomorphic to
\[
C_2 \times C_2.
\]

\subsection*{Step 5: Solvability by Radicals}

Since the Galois group is abelian, it is solvable, confirming that the polynomial is solvable by radicals.

\section*{Example 4: Quintic Polynomial}

Consider the quintic polynomial
\[
x^5 - x - 1 = 0.
\]

\subsection*{Step 1: Irreducibility}

The polynomial is irreducible over \(\mathbb{Q}\), for example by Eisenstein’s criterion or modular reduction.

\subsection*{Step 2: Discriminant}

The discriminant is nonzero and not a perfect square.

\subsection*{Step 3: Galois Group}

Computational and theoretical analysis shows that the Galois group is
\[
S_5.
\]

\subsection*{Step 4: Solvability by Radicals}

Since \(S_5\) is not solvable, the general quintic polynomial cannot be solved by radicals. These examples illustrate how the roots of a polynomial determine its splitting field, how automorphisms form the Galois group, and how the structure of this group indicates whether the polynomial is solvable by radicals. Quadratic, cubic, and quartic polynomials have solvable Galois groups, while a generic quintic has a non-solvable group, confirming the classical limitation discovered by Galois theory.	
	
	
	\section{Implementation}
	
	A graphical computational tool was developed using Python and the Kivy library. The program accepts polynomial coefficients, computes roots numerically, evaluates discriminants, and suggests likely Galois group classifications.
	\begin{figure}[h!]
		\centering
		\includegraphics[width=0.8\textwidth]{pd1.JPG}
		\caption{Front Page of the Mobile App}
		\label{fig:benue_map1.png}
	\end{figure}
	\begin{figure}[h!]
		\centering
		\includegraphics[width=0.8\textwidth]{pd2.JPG}
		\caption{First Page of the Mobile App}
		\label{fig:benue_map2.png}
	\end{figure}
	\begin{figure}[h!]
		\centering
		\includegraphics[width=0.8\textwidth]{pd3.JPG}
		\caption{Calculation Page of the Mobile App}
		\label{fig:benue_map3.png}
	\end{figure}
	\chapter{Conclusion}	This project demonstrates how Galois theory transforms polynomial analysis into the study of symmetry. By combining algebraic theory with computational tools, it becomes possible to analyze polynomial structures efficiently and visualize abstract concepts. The computational approach enhances both research and teaching by making Galois theory more accessible and interactive.
\end{document}